\documentclass[12pt]{article}

\usepackage{cmap}
\usepackage[T1,T2A]{fontenc}
\usepackage[utf8]{inputenc}
\usepackage[english, russian]{babel}
\usepackage{amssymb}
\usepackage{amsmath}
\usepackage{amsthm}
\usepackage{dsfont}
\usepackage{bm}
\usepackage{diagbox}
\usepackage[left=20mm,right=10mm,top=20mm,bottom=20mm,bindingoffset=2mm]{geometry}
\usepackage{indentfirst}
\usepackage{float}
\usepackage[hidelinks]{hyperref}
\usepackage{xcolor}
\usepackage{listings}

\DeclareMathOperator{\N}{\mathbb{N}}
\DeclareMathOperator{\R}{\mathbb{R}}
\DeclareMathOperator{\Z}{\mathbb{Z}}
\DeclareMathOperator{\CC}{\mathbb{C}}
\DeclareMathOperator{\PP}{\mathrm{P}}
\DeclareMathOperator{\Expec}{\mathrm{E}}
\DeclareMathOperator{\Var}{\mathrm{Var}}
\DeclareMathOperator{\Cov}{\mathrm{Cov}}
\DeclareMathOperator{\asConv}{\xrightarrow{a.s.}}
\DeclareMathOperator{\LpConv}{\xrightarrow{Lp}}
\DeclareMathOperator{\pConv}{\xrightarrow{p}}
\DeclareMathOperator{\dConv}{\xrightarrow{d}}

\hypersetup{
    colorlinks=true,
    linkcolor=blue,
    citecolor=blue,
    urlcolor=blue
}

\addto\captionsrussian{\renewcommand{\refname}{Список использованных источников}}

\begin{document}

\begin{titlepage}
    \begin{center}
        \large{Федеральное государственное автономное образовательное учреждение высшего образования <<Национальный исследовательский университет ИТМО>>}
    \end{center}
    
    \vspace{15em}
    
    \begin{center}
        \huge{\textbf{Расчётно-графическая работа №1}} \\
        \large{По дисциплине <<Математическая статистика>>}
    \end{center}
    
    \vspace{5em}
    
    \begin{flushright}
        \Large{\textbf{Михайлов Дмитрий Андреевич}} \\
        \Large{P3206} \\
        \Large{368530} \\
        \Large{\textbf{Медведев Владислав Александрович}} \\
        \Large{P3206} \\
        \Large{368508}
    \end{flushright}
    
    \vspace{10em}
    
    \begin{center}
        Санкт-Петербург \\
        2025 год
    \end{center}
\end{titlepage}

\tableofcontents
\newpage

\addcontentsline{toc}{section}{Задача №1}
\section*{Задача №1}

\textbf{Условие задачи.}

В файле \href{https://drive.google.com/file/d/1vv2jGNp6EO8HHRoscDRQU90faR3j8iTN/view}{cars93.csv} представлены данные об автомобилях, проданных в некотором автосалоне за 93 год. Какие типы автомобилей представлены в датасете? Какой тип наиболее распространён, какой - менее? Рассчитайте выборочное среднее, выборочную дисперсию, выборочную медиану и межквартильный размах мощности для всей совокупности автомобилей и отдельно для американских и не американских авто. Построить график эмпирической функции распределения, гистограмму и box-plot мощности для всей совокупности и отдельно для каждого типа авто.

\vspace{1em}

\addcontentsline{toc}{section}{Задача №2}
\section*{Задача №2}

\textbf{Условие задачи.}

Предположите, какому вероятностному закону соответствует рапределения показателя, рассмотренного (расчёт выборочных характеристик и визуализация) в задании №1. Оцените параметры данного распределения методом максимального правдоподобия или методом моментов (\textbf{математическое обоснование оценки строго обязательно}). Какими статистическими свойствами обладает найденная оценка (\textbf{обосновать})? Найти \textbf{теоретические} смещение, дисперсию, MSE (или хотя бы написать теоретические формулы, по которым данные показатели вычисляются, если в итоге получается <<очень сложный>> интеграл/ряд), информацию Фишера (если определена для вашей модели).

\vspace{1em}

\textbf{Решение.}

Решение этой задачи можно представить аналитически. На основе представленных в первом задании данных можно заметить, что значение мощности автомобиля варьируется в определённом диапазоне, не мало важно заметить следующий факт, что значения мощностей автомобилей распределены симметрично вокруг среднего. Соответственно, принимая мощность автомобиля за случайную величину, можно предположить, что данная случайная величина имеет нормальное распределение. 
\begin{gather}
    \label{LawOfNormalDistribution}
    \ X \sim N(\mu, \sigma^2)\
\end{gather}

Следовательно, оценивая параметры $\theta_1, \theta_2$ методом моментов для нормального распределения, вычисляем сперва теоретические моменты $E[X], E[X^2]$. \\
\begin{gather*}
    \begin{cases}
    \ E[X] = \mu \\
    E[X^2] = \mu^2 + \sigma^2\
    \end{cases}
    \label{TheoreticalMoments}
\end{gather*}

Далее, нам необходимо оценить выборочные моменты $ \hat{m}_1, \hat{m}_2 $.
\begin{gather*}
    \begin{cases}
    \ \hat{m}_1 = \frac{1}{n} \sum X_i \\
    \hat{m}_2 = \frac{1}{n} \sum X_i^2\
    \end{cases}
    \label{SelectiveMoments}
\end{gather*}

В конечном итоге, после приравния получаем следующую систему, откуда можно найти оценки наших параметров нормального распределения \eqref{LawOfNormalDistribution}:
\begin{gather*}
    \begin{cases}
    \ \hat{\mu} = \hat{m}_1 \\
    \ \hat{\sigma}^2 = \hat{m}_2 - \hat{m}_1^2\
    \end{cases}
    \label{FinalSystemOfEquations}
\end{gather*}
\newpage

Таким образом, оценки параметров будут следующие:
\begin{gather*}
    \begin{cases}
    \ \hat{\mu} = \bar{X} \\
    \ \hat{\sigma}^2 = S^2\
    \end{cases}
    \label{FinalSystemOfEquations}
\end{gather*}

Проверим свойство несмещенности у полученных оценок:
\begin{gather*}
    \begin{cases}
    \  E[\hat{\mu}] = \mu \\
    \ E[\hat{\sigma}^2] = \sigma^2\
    \end{cases}
\end{gather*}

После того, как мы убедились в наличии этого свойства, так же стоит проверить на наличие свойства дисперсии оценок:
\begin{gather*}
    \begin{cases}
    \ Var(\hat{\mu}) = \frac{\sigma^2}{n} \\
    \ Var(\hat{\sigma}^2) = \frac{2\sigma^4}{n}\
    \end{cases}
\end{gather*}

Последняя проверка на наличие свойства о средней квадратичной ошибке:
\begin{gather}
    \label{RootMeanSquareError1}
    \ E\Vert\hat{\theta} - \theta\Vert^2 = Var({\hat{\theta}}) + \Vert bias(\hat{\theta})\Vert^2\
\end{gather}

Или же:
\begin{gather}
    \label{RootMeanSquareError2}
    \ MSE(\hat{\theta}) = Var({\hat{\theta}}) + \Vert bias(\hat{\theta})\Vert^2\
\end{gather}

Зная, что для несмещённой оценки $\Vert bias(\hat{\theta})\Vert^2 = 0$, получаем итоговую систему для наших оценок:
\begin{gather*}
    \begin{cases}
        \ MSE(\hat{\mu}) = \frac{\sigma^2}{n} \\ \
        \ MSE(\hat{\sigma}^2) = \frac{2\sigma^4}{n}
    \end{cases}
\end{gather*}
\newpage

\addcontentsline{toc}{section}{Задача №3}
\section*{Задача №3}

\textbf{Условие задачи.}

Пусть $P_{\theta}$ - выбранное в предыдущем задании распределение, параметризующееся вектором $\theta$ (пример - равномерное распределение на $[-2\theta; 4\theta], \hat{\theta} = \bar{X}$), $\hat{\theta}$ - оценка параметра $\theta$, полученная в предыдущем упражнении. Проведите численный эксперимент по следующей схеме:
\begin{itemize}
    \item Зафиксируйте конкретное значение $\theta=\theta_0$
    \item Заведите массив ${n_1, ..., n_k}$ объёмов выборки
    \item Сгенерируйте из распределения $P_{\theta_0}$ достаточно большое количество $M$ выборок объёма $n$, где $n$ принимает значения из массива ${n_1, ..., n_m}$. Для каждой сгенерированной выборки вычислите оценку $\hat{\theta}$.
    \item Эмперически рассмотреть поведение оценки $\hat{\theta}$ в зависимости от объёма выборки (можно для каждого объёма выборки $n_i$ вывести описательные статистики для оценок, изобразить гистограмму, box-plot, violin-plot).
\end{itemize}
\newpage
\addcontentsline{toc}{section}{Приложения}
\section*{Приложения}

\subsection*{Задача №1}

Ссылка на исходник с кодом программы, решающей эту задачу на языке Python. \cite{TaskNumber1}

\subsection*{Задача №3}
Ссылка на исходник с кодом программы, решающей эту задачу на языке Python. \cite{TaskNumber3}
\newpage

\addcontentsline{toc}{section}{Список использованных источников}
\begin{thebibliography}{99}
    \bibitem{TaskNumber1}
    Задача №1. \textit{URL}: \href{https://colab.research.google.com/drive/10P1FfFSNAT1sFofKclQP3D2H63IfjJ5n#scrollTo=sdMVrQmwrOrM}{Исходник с кодом, решающий задачу №1.}
    
    \bibitem{TaskNumber3}
    Задача №3. \textit{URL}:
\href{https://colab.research.google.com/drive/1oKAvJfOlzvyp6hB6qVOLeUKA08yHvjOA?usp=sharing#scrollTo=IJjFoXfdsgKB}{Исходник с кодом, решающий задачу №3.}
\end{thebibliography}

\end{document}
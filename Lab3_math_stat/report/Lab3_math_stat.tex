\documentclass[12pt]{article}

\usepackage{cmap}
\usepackage[T1,T2A]{fontenc}
\usepackage[utf8]{inputenc}
\usepackage[english, russian]{babel}
\usepackage{amssymb}
\usepackage{amsmath}
\usepackage{amsthm}
\usepackage{dsfont}
\usepackage{bm}
\usepackage{diagbox}
\usepackage[left=20mm,right=10mm,top=20mm,bottom=20mm,bindingoffset=2mm]{geometry}
\usepackage{indentfirst}
\usepackage{float}
\usepackage[hidelinks]{hyperref}
\usepackage{xcolor}
\usepackage{listings}

\DeclareMathOperator{\N}{\mathbb{N}}
\DeclareMathOperator{\R}{\mathbb{R}}
\DeclareMathOperator{\Z}{\mathbb{Z}}
\DeclareMathOperator{\CC}{\mathbb{C}}
\DeclareMathOperator{\PP}{\mathrm{P}}
\DeclareMathOperator{\Expec}{\mathrm{E}}
\DeclareMathOperator{\Var}{\mathrm{Var}}
\DeclareMathOperator{\Cov}{\mathrm{Cov}}
\DeclareMathOperator{\asConv}{\xrightarrow{a.s.}}
\DeclareMathOperator{\LpConv}{\xrightarrow{Lp}}
\DeclareMathOperator{\pConv}{\xrightarrow{p}}
\DeclareMathOperator{\dConv}{\xrightarrow{d}}

\hypersetup{
	colorlinks=true,
	linkcolor=blue,
	citecolor=blue,
	urlcolor=blue
}

\addto\captionsrussian{\renewcommand{\refname}{Список использованных источников}}

\begin{document}
	
	\begin{titlepage}
		\begin{center}
			\large{Федеральное государственное автономное образовательное учреждение высшего образования <<Национальный исследовательский университет ИТМО>>}
		\end{center}
		
		\vspace{15em}
		
		\begin{center}
			\huge{\textbf{Расчётно-графическая работа №3}} \\
			\large{По дисциплине <<Математическая статистика>>}
		\end{center}
		
		\vspace{5em}
		
		\begin{flushright}
			\Large{\textbf{Михайлов Дмитрий Андреевич}} \\
			\Large{P3206} \\
			\Large{368530} \\
			\Large{\textbf{Медведев Владислав Александрович}} \\
			\Large{P3206} \\
			\Large{368508}
		\end{flushright}
		
		\vspace{10em}
		
		\begin{center}
			Санкт-Петербург \\
			2025 год
		\end{center}
	\end{titlepage}
	
	\tableofcontents
	\newpage
	
	\addcontentsline{toc}{section}{Задача №1}
	\section*{Задача №1}
	
	\textbf{Условие задачи.}
	
	Для каждой проблемы нужно провести два статистических теста, если не сказано иное, причём первый из критериев нужно реализовать самостоятельно (считать и выводить значение статистики, критическое значение, p-value), в качестве второго можно воспользоваться готовой реализацией. Также нужно отдельно указывать, как формализуются $H_0$ и $H_1$ для выбранных тестов. Уровень значимости выбираете сами.
	
	\textbf{Вариант 1}
	
	В файле \href{https://drive.google.com/file/d/1cx0pshptDSVmaWLJCBGS9jIIJ2g-VRgT/view}{kc\_house\_data.csv} приведены данные о цене на недвижимость где-то в окрестности Сиэтла.
	
	\begin{enumerate}
		\item Предположите с каким вероятностным законом распределена цена. С помощью статистического теста подтвердите/опровергните это предположение (первый тест - критерий согласия Колмогорова, если распределение абсолютно непрерывное, либо критерий согласия Пирсона хи-квадрат, если распределение дискретное).
		
		\item Верно ли, что цена на старый и новый фонд распределена одинаково (порог возраста выбирайте сами) (первый тест - критерий однородности Смирнова или хи-квадрат, или f-тест + t-тест)?
		
		\item Верно ли, что при увеличении \textquotedblleft жилищной площади\textquotedblright 	растёт и цена (первый тест - критерий на один из коэффициентов корреляции)?
	\end{enumerate}
	
	\textbf{Решение.}
	
	Критерий согласия Колмогорова:
	\begin{gather*}
		\begin{cases}
			\ H_0 = F(x) = F_0(x), \text{(данные имеют заданное теоретическое распределение)} \\
			\ H_1 = F(x) \ne F_0(x), \text{(данные НЕ имеют заданное распределение)} \
		\end{cases}
	\end{gather*}
	
	где:
	\begin{itemize}
		\item $ F(x) $ — эмпирическая функция распределения
		\item $ F_0(x) $ — теоретическая функция распределения
	\end{itemize}
	
	Функция распределения Колмогорова: $$ P(\sqrt{n} D_n \le \lambda) \to 1 - 2 \sum_{k=1}^{\infty} (-1)^{k-1} e^{-2k^2 \lambda^2} $$
	
	Тогда: $$ P(\sqrt{n} D_n > \lambda) \approx 2 \sum_{k=1}^{\infty} (-1)^{k-1} e^{-2k^2 \lambda^2} $$
	
	Это и есть p-value - вероятность того, что наблюдаемая статистика $ D_n $ больше полученной.
	\newpage
	
	Критерий согласия Андерсона-Дарлинга:
	
	Для отсортированной выборки $ x_1 \le x_2 \le \dots \le x_n $, и теоретической CDF $ F(x) $:
	
	$$ A^2 = -n - \frac{1}{n} \sum_{i=1}^{n} \left[ (2i - 1) \cdot \left( \ln F(x_i) + \ln (1 - F(x_{n+1-i})) \right) \right] $$
	
	где:
	\begin{itemize}
		\item $ F(x) $ — функция распределения предполагаемого закона (например, нормального)
		\item $ n $ — размер выборки
	\end{itemize}
	\vspace*{1em}
	
	Проверим гипотезы с помощью статистических тестов:
	\vspace*{1em}
	
	\textbf{Критерий Смирнова} — сравнение распределений.
	
	\textbf{Критерий хи-квадрат} — сравнения фактических данных в выборке с теоретическими результатами.
	
	Возьмем за порог 1980 год.
	\vspace*{3em}
	
	\textbf{KS-тест (критерий Колмогорова-Смирнова)}
	
	Цель: сравнить два эмпирических распределения $ F_n(x) $ и $ G_m(x) $
	
	$$ H_0: F(x) = G(x), \text{(распределения одинаковы)} $$
	
	\textbf{Статистика Колмогорова: }
	
	$$ D = \sup_x |F_n(x) - G_m(x)| $$
	
	где:
	\begin{itemize}
		\item $ F_n(x) $ — эмпирическая функция распределения первой выборки
		\item $ G_m(x) $ — эмпирическая функция второй выборки
		\item $ \sup $ — супремум (максимальное расстояние между функциями)
	\end{itemize}
	\vspace*{1em}
	
	\textbf{Критерий хи-квадрат (на однородность)}
	
	Критерий однородности $\chi^2$ используется для проверки, одинаково ли распределяется признак (в нашем случае — цена по квартилям) в двух или более группах (старый и новый фонд жилья).
	
	Он сравнивает наблюдаемые частоты с ожидаемыми, которые рассчитываются при условии, что распределения в группах одинаковые.
	
	$$ \chi^2 = \sum_{i=1}^{k} \sum_{j=1}^{m} \frac{(O_{ij} - E_{ij})^2}{E_{ij}} $$
	
	где:
	\begin{itemize}
		\item $ O_{ij} $ — наблюдаемое количество объектов в ячейке (i, j),
		\item $ E_{ij} $ — ожидаемое количество объектов в ячейке (i, j), вычисляемое по формуле: $$ E_{ij} = \frac{(\text{сумма по строке i}) \cdot (\text{сумма по столбцу j})}{\text{общая сумма}} $$
		\item $ k = 2 $ — количество групп (старый и новый фонд)
		\item $ m = 4 $ — количество интервалов (квартильные группы)
	\end{itemize}
	\vspace*{3em}
	
	\textbf{Порядок действий: }
	\begin{enumerate}
		\item Разделить выборку по году постройки на старый и новый фонд.
		
		\item Определить квартильные границы по всей совокупности цен.
		
		\item Отнести каждое наблюдение к одному из квартилей.
		
		\item Построить таблицу частот: по фондам и квартилям.
		
		\item Вычислить статистику $\chi^2$.
		
		\item Сравнить результат с критическим значением $\chi^2$ для уровня значимости (например, $\alpha$ = 0.05) и нужного числа степеней свободы:  
		$$ df = (k - 1)(m - 1) = (2 - 1)(4 - 1) = 3 $$
	\end{enumerate}
	Мы используем именно критерий однородности, а не независимости или согласия, потому что сравниваем распределения в разных группах, а не проверяем связь между двумя признаками.
	\vspace*{3em}
	
	Ожидаемые частоты считаются так:
	
	$$ E_{ij} = \frac{\text{сумма по строке i} \times \text{сумма по столбцу j}}{\text{общая сумма}} $$
	
	\textbf{Статистика $\chi^2$}
	
	После этого мы считаем:
	
	$$ \chi^2 = \sum_{i,j} \frac{(O_{ij} - E_{ij})^2}{E_{ij}} $$
	
	В результате для наших данных получается:
	
	$$ \chi^2 \approx 444.91 $$
	
	\textbf{Степени свободы} считаются по формуле:
	
	$$ df = (r-1) \times (c-1) $$
	
	где:
	\begin{itemize}
		\item $ r $ — количество строк в таблице (у нас 2 группы: старый и новый фонд)
		\item $ c $ — количество столбцов (у нас 4 квартиля)
	\end{itemize}
	
	Подставляем:
	
	$$ df = (2-1) \times (4-1) = 1 \times 3 = 3 $$
	
	Итого $ df = 3 $.
	
	Когда известно $\chi^2$ и степени свободы $df$, p-значение — это вероятность получить значение статистики ещё большее, чем наблюдаемое, при справедливости нулевой гипотезы.
	
	p-значение считается через распределение хи-квадрат:
	
	$$ p = P(\chi^2 > 444.91) $$
	
	$$ p = 4.13 \times 10^{-96} $$
	
	То есть вероятность случайно получить такие большие различия практически нулевая.
	\vspace*{1em}
	
	\textbf{Вывод.}
	
	p-значение $(\chi^2$) < 0.05 → распределение отличается
	
	p-значение KS < 0.05 → Форма распределения цен также различна.
	\newpage
	
	\textbf{Критерий на коэффициент корреляции Пирсона.}
	
	Гипотезы.
	\begin{gather*}
		\begin{cases}
			\ H_0: \rho = 0 \quad \text{(нет корреляции)} \\
			\ H_1: \rho \neq 0 \quad \text{(есть корреляция)}
		\end{cases}
	\end{gather*}
	
	Здесь $\rho$ — истинный коэффициент корреляции Пирсона в генеральной совокупности.
	\vspace*{1em}
	
	\textbf{Коэффициент корреляции Пирсона.}
	
	Оценка $ r $ вычисляется по формуле:
	
	$$ r = \frac{\sum_{i=1}^{n} (x_i - \bar{x})(y_i - \bar{y})}{\sqrt{\sum_{i=1}^{n} (x_i - \bar{x})^2} \cdot \sqrt{\sum_{i=1}^{n} (y_i - \bar{y})^2}} $$
	
	где $ \bar{x} = \frac{1}{n} \sum x_i $, $ \bar{y} = \frac{1}{n} \sum y_i $ — выборочные средние.
	\vspace*{1em}
	
	\textbf{Статистика критерия.}
	
	Если нулевая гипотеза верна и $ \rho = 0 $, то статистика:
	
	$$ t = \frac{r \sqrt{n - 2}}{\sqrt{1 - r^2}} $$
	
	распределена по t-распределению Стьюдента с $n - 2 $ степенями свободы.
	\vspace*{1em}
	
	\textbf{Правило принятия решения.}
	
	\begin{enumerate}
		\item Найдём критическое значение $ t_{\text{crit}} $ для уровня значимости $ \alpha $ и $ n - 2 $ степеней свободы:
		
		$$ t_{\text{crit}} = t_{1 - \alpha / 2}(n - 2) $$
		
		\item Вычислим двустороннее значение p-value:
		
		$$ p = 2 \cdot P(T > |t_{\text{набл}}|) $$
		
		\item Если $ p < \alpha $, то отвергаем $ H_0 $ в пользу $ H_1 $.
	\end{enumerate}
	\vspace*{1em}
	
	\textbf{Критерий на коэффициент корреляции Спирмена.}
	\vspace*{1em}
	
	\textbf{Оценка коэффициента Спирмена.}
	
	Коэффициент Спирмена $ r_s $ вычисляется по формуле:
	
	$$ r_s = 1 - \frac{6 \sum_{i=1}^{n} d_i^2}{n(n^2 - 1)} $$
	
	где $ d_i = \text{rg}(x_i) - \text{rg}(y_i) $ — разность рангов соответствующих элементов выборки.
	(Ранг — это порядковый номер элемента в отсортированной выборке)
	
	Или, как в `scipy`, с помощью корреляции Пирсона между рангами:
	
	$$ r_s = \text{corr}(\text{rg}(x), \text{rg}(y)) $$
	\vspace*{1em}
	
	\textbf{Приближённая t-статистика.}
	
	Для больших $ n $ приближённо применяется t-распределение с $ n - 2 $ степенями свободы:
	
	$$ t = \frac{r_s \cdot \sqrt{n - 2}}{\sqrt{1 - r_s^2}} \sim t_{n-2} $$
	\vspace*{1em}
	
	\textbf{Критическое значение и p-value.}
	
	\begin{itemize}
		\item Критическое значение:
		$t_{\text{crit}} = t_{1 - \alpha / 2}(n - 2)$
		\item Двусторонний p-value:
		$p = 2 \cdot P(T > |t| \mid H_0)$
	\end{itemize}
	\vspace*{1em}
	
	\textbf{Правило принятия решения.}
	
	\begin{itemize}
		\item Если $ |t| > t\_{\text{crit}} $, то отвергаем $ H\_0 $ — есть статистически значимая корреляция.
		\item Иначе — не отвергаем $ H_0 $ — доказательств корреляции недостаточно.
	\end{itemize}
	
	Далее приведён код на языке Python.
	\addcontentsline{toc}{section}{Приложения}
	\section*{Приложения}
	
	\subsection*{Задача №1}
	
	Ссылка на исходник с кодом программы, решающей эту задачу на языке Python. \cite{TaskNumber1}
	\newpage
	
	\addcontentsline{toc}{section}{Список использованных источников}
	\begin{thebibliography}{99}
		\bibitem{TaskNumber1}
		Задача №1. \textit{URL}: \href{https://colab.research.google.com/drive/1ViTlo3eksmFZN0RmQYRjO1QAUtpfM8Ru?usp=sharing}{Исходник с кодом, решающий задачу №1.}
	\end{thebibliography}
	
\end{document}
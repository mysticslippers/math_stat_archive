\documentclass[12pt]{article}

\usepackage{cmap}
\usepackage[T1,T2A]{fontenc}
\usepackage[utf8]{inputenc}
\usepackage[english, russian]{babel}
\usepackage{amssymb}
\usepackage{amsmath}
\usepackage{amsthm}
\usepackage{dsfont}
\usepackage{bm}
\usepackage{diagbox}
\usepackage[left=20mm,right=10mm,top=20mm,bottom=20mm,bindingoffset=2mm]{geometry}
\usepackage{indentfirst}
\usepackage{float}
\usepackage[hidelinks]{hyperref}
\usepackage{xcolor}
\usepackage{listings}

\DeclareMathOperator{\N}{\mathbb{N}}
\DeclareMathOperator{\R}{\mathbb{R}}
\DeclareMathOperator{\Z}{\mathbb{Z}}
\DeclareMathOperator{\CC}{\mathbb{C}}
\DeclareMathOperator{\PP}{\mathrm{P}}
\DeclareMathOperator{\Expec}{\mathrm{E}}
\DeclareMathOperator{\Var}{\mathrm{Var}}
\DeclareMathOperator{\Cov}{\mathrm{Cov}}
\DeclareMathOperator{\asConv}{\xrightarrow{a.s.}}
\DeclareMathOperator{\LpConv}{\xrightarrow{Lp}}
\DeclareMathOperator{\pConv}{\xrightarrow{p}}
\DeclareMathOperator{\dConv}{\xrightarrow{d}}

\hypersetup{
	colorlinks=true,
	linkcolor=blue,
	citecolor=blue,
	urlcolor=blue
}

\addto\captionsrussian{\renewcommand{\refname}{Список использованных источников}}

\begin{document}
	
	\begin{titlepage}
		\begin{center}
			\large{Федеральное государственное автономное образовательное учреждение высшего образования <<Национальный исследовательский университет ИТМО>>}
		\end{center}
		
		\vspace{15em}
		
		\begin{center}
			\huge{\textbf{Расчётно-графическая работа №3}} \\
			\large{По дисциплине <<Математическая статистика>>}
		\end{center}
		
		\vspace{5em}
		
		\begin{flushright}
			\Large{\textbf{Михайлов Дмитрий Андреевич}} \\
			\Large{P3206} \\
			\Large{368530} \\
			\Large{\textbf{Медведев Владислав Александрович}} \\
			\Large{P3206} \\
			\Large{368508}
		\end{flushright}
		
		\vspace{10em}
		
		\begin{center}
			Санкт-Петербург \\
			2025 год
		\end{center}
	\end{titlepage}
	
	\tableofcontents
	\newpage
	
	\addcontentsline{toc}{section}{Задача №1}
	\section*{Задача №1}
	
	\textbf{Условие задачи.}
	
	Для каждой проблемы нужно провести два статистических теста, если не сказано иное, причём первый из критериев нужно реализовать самостоятельно (считать и выводить значение статистики, критическое значение, p-value), в качестве второго можно воспользоваться готовой реализацией. Также нужно отдельно указывать, как формализуются $H_0$ и $H_1$ для выбранных тестов. Уровень значимости выбираете сами.
	
	\textbf{Вариант 1}
	
	В файле \href{https://drive.google.com/file/d/1cx0pshptDSVmaWLJCBGS9jIIJ2g-VRgT/view}{kc\_house\_data.csv} приведены данные о цене на недвижимость где-то в окрестности Сиэтла.
	
	\begin{enumerate}
		\item Предположите с каким вероятностным законом распределена цена. С помощью статистического теста подтвердите/опровергните это предположение (первый тест - критерий согласия Колмогорова, если распределение абсолютно непрерывное, либо критерий согласия Пирсона хи-квадрат, если распределение дискретное).
		
		\item Верно ли, что цена на старый и новый фонд распределена одинаково (порог возраста выбирайте сами) (первый тест - критерий однородности Смирнова или хи-квадрат, или f-тест + t-тест)?
		
		\item Верно ли, что при увеличении \textquotedblleft жилищной площади\textquotedblright 	растёт и цена (первый тест - критерий на один из коэффициентов корреляции)?
	\end{enumerate}
	
	\textbf{Решение.}
	
	Решение представлено на языке Python.
	\vspace*{1em}
	\newpage
	
	\addcontentsline{toc}{section}{Приложения}
	\section*{Приложения}
	
	\subsection*{Задача №1}
	
	Ссылка на исходник с кодом программы, решающей эту задачу на языке Python. \cite{TaskNumber1}
	\newpage
	
	\addcontentsline{toc}{section}{Список использованных источников}
	\begin{thebibliography}{99}
		\bibitem{TaskNumber1}
		Задача №1. \textit{URL}: \href{https://colab.research.google.com/drive/1ViTlo3eksmFZN0RmQYRjO1QAUtpfM8Ru?usp=sharing}{Исходник с кодом, решающий задачу №1.}
	\end{thebibliography}
	
\end{document}
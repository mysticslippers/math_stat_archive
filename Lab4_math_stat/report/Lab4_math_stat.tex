\documentclass[12pt]{article}

\usepackage{cmap}
\usepackage[T1,T2A]{fontenc}
\usepackage[utf8]{inputenc}
\usepackage[english, russian]{babel}
\usepackage{amssymb}
\usepackage{amsmath}
\usepackage{amsthm}
\usepackage{dsfont}
\usepackage{bm}
\usepackage{diagbox}
\usepackage[left=20mm,right=10mm,top=20mm,bottom=20mm,bindingoffset=2mm]{geometry}
\usepackage{indentfirst}
\usepackage{float}
\usepackage[hidelinks]{hyperref}
\usepackage{xcolor}
\usepackage{listings}

\DeclareMathOperator{\N}{\mathbb{N}}
\DeclareMathOperator{\R}{\mathbb{R}}
\DeclareMathOperator{\Z}{\mathbb{Z}}
\DeclareMathOperator{\CC}{\mathbb{C}}
\DeclareMathOperator{\PP}{\mathrm{P}}
\DeclareMathOperator{\Expec}{\mathrm{E}}
\DeclareMathOperator{\Var}{\mathrm{Var}}
\DeclareMathOperator{\Cov}{\mathrm{Cov}}
\DeclareMathOperator{\asConv}{\xrightarrow{a.s.}}
\DeclareMathOperator{\LpConv}{\xrightarrow{Lp}}
\DeclareMathOperator{\pConv}{\xrightarrow{p}}
\DeclareMathOperator{\dConv}{\xrightarrow{d}}

\hypersetup{
	colorlinks=true,
	linkcolor=blue,
	citecolor=blue,
	urlcolor=blue
}

\addto\captionsrussian{\renewcommand{\refname}{Список использованных источников}}

\begin{document}
	
	\begin{titlepage}
		\begin{center}
			\large{Федеральное государственное автономное образовательное учреждение высшего образования <<Национальный исследовательский университет ИТМО>>}
		\end{center}
		
		\vspace{15em}
		
		\begin{center}
			\huge{\textbf{Расчётно-графическая работа №4}} \\
			\large{По дисциплине <<Математическая статистика>>}
		\end{center}
		
		\vspace{5em}
		
		\begin{flushright}
			\Large{\textbf{Михайлов Дмитрий Андреевич}} \\
			\Large{P3206} \\
			\Large{368530} \\
			\Large{\textbf{Медведев Владислав Александрович}} \\
			\Large{P3206} \\
			\Large{368508}
		\end{flushright}
		
		\vspace{10em}
		
		\begin{center}
			Санкт-Петербург \\
			2025 год
		\end{center}
	\end{titlepage}
	
	\tableofcontents
	\newpage
	
	\addcontentsline{toc}{section}{Задача №1}
	\section*{Задача №1}
	
	\textbf{Условие задачи.}
	
	Задание представлено в 4 вариантах. Для каждого варианта требуется построить линейную модель (предполагая нормальность распределения ошибок, некоррелированность компонент, гомоскедастичность), вычислить оценки коэффициентов модели и остаточной дисперсии, построить для них доверительные интервалы, вычислить коэффициент детерминации, проверить указанные в условии гипотезы с помощью построенной линейной модели.
	
	\textbf{Указание}: из встроенных функций разрешается пользоваться квантильными функциями и средствами для квадратичной оптимизации (иными словами, готовую обертку для построения линейной модели не использовать, максимум можете сравнить вашу реализацию с готовой)
	
	\textbf{Вариант 1}.
	В файле \href{https://drive.google.com/file/d/1vv2jGNp6EO8HHRoscDRQU90faR3j8iTN/view}{cars93.csv} представлены данные о продажах различных авто.
	
	\begin{enumerate}
		\item Постройте линейную модель, где в качестве независимых переменных выступают расход в городе, расход на шоссе, мощность (вместе со свободным коэффициентом), зависимой - цена.
		
		\item Проверьте следующие подозрения:
		\begin{itemize}
			\item Чем больше мощность, тем больше цена
			
			\item Цена изменяется в зависимости от расхода в городе
			
			\item Проверьте гипотезу $H_0$ о равенстве одновременно нулю коэффициентов при расходе в городе и расходе на шоссе против альтернативы $H_1 = \bar{H_0}$
		\end{itemize}
	\end{enumerate}
	
	\textbf{Решение.}
	
	Решение представлено на языке Python.
	\vspace*{1em}
	
	\addcontentsline{toc}{section}{Задача №2}
	\section*{Задача №2}
	
	\textbf{Условие задачи.}
	
	Для каждого варианта требуется проверить гипотезу о равенстве средних на каждом уровне фактора с помощью модели однофакторного дисперсионного анализа.
	
	\textbf{Указание}: реализовать самим.
	
	\textbf{Вариант 2.}
	В файле \href{https://drive.google.com/file/d/14L_y0LOAebuuqh8PllOw64cJQwVkmlV6/view}{exams\_dataset.csv} представлены данные о сдаче экзаменов. Фактор - этническая/национальная группа. Выходная переменная - суммарный балл за все три экзамена.
	\vspace*{1em}
	
	\textbf{Решение.}
	
	Решение представлено на языке Python.
	\newpage
	
	\addcontentsline{toc}{section}{Приложения}
	\section*{Приложения}
	
	\subsection*{Задача №1}
	
	Ссылка на исходник с кодом программы, решающей эту задачу на языке Python. \cite{TaskNumber1}
	
	\subsection*{Задача №2}
	
	Ссылка на исходник с кодом программы, решающей эту задачу на языке Python. \cite{TaskNumber2}
	\newpage
	
	\addcontentsline{toc}{section}{Список использованных источников}
	\begin{thebibliography}{99}
		\bibitem{TaskNumber1}
		Задача №1. \textit{URL}: \href{https://colab.research.google.com/drive/1hU3V-uFTl9Eqq9hKGlbg4017R-LNO2Ed?usp=sharing}{Исходник с кодом, решающий задачу №1.}
		\bibitem{TaskNumber2}
		Задача №2. \textit{URL}: \href{https://colab.research.google.com/drive/1jZ2odN8TAdEDsBw6dNhFYbSYFFkQi2u0?usp=sharing}{Исходник с кодом, решающий задачу №2.}
	\end{thebibliography}
	
\end{document}
\documentclass[12pt]{article}

\usepackage{cmap}
\usepackage[T1,T2A]{fontenc}
\usepackage[utf8]{inputenc}
\usepackage[english, russian]{babel}
\usepackage{amssymb}
\usepackage{amsmath}
\usepackage{amsthm}
\usepackage{dsfont}
\usepackage{bm}
\usepackage{diagbox}
\usepackage[left=20mm,right=10mm,top=20mm,bottom=20mm,bindingoffset=2mm]{geometry}
\usepackage{indentfirst}
\usepackage{float}
\usepackage[hidelinks]{hyperref}
\usepackage{xcolor}
\usepackage{listings}

\DeclareMathOperator{\N}{\mathbb{N}}
\DeclareMathOperator{\R}{\mathbb{R}}
\DeclareMathOperator{\Z}{\mathbb{Z}}
\DeclareMathOperator{\CC}{\mathbb{C}}
\DeclareMathOperator{\PP}{\mathrm{P}}
\DeclareMathOperator{\Expec}{\mathrm{E}}
\DeclareMathOperator{\Var}{\mathrm{Var}}
\DeclareMathOperator{\Cov}{\mathrm{Cov}}
\DeclareMathOperator{\asConv}{\xrightarrow{a.s.}}
\DeclareMathOperator{\LpConv}{\xrightarrow{Lp}}
\DeclareMathOperator{\pConv}{\xrightarrow{p}}
\DeclareMathOperator{\dConv}{\xrightarrow{d}}

\hypersetup{
	colorlinks=true,
	linkcolor=blue,
	citecolor=blue,
	urlcolor=blue
}

\addto\captionsrussian{\renewcommand{\refname}{Список использованных источников}}

\begin{document}
	
	\begin{titlepage}
		\begin{center}
			\large{Федеральное государственное автономное образовательное учреждение высшего образования <<Национальный исследовательский университет ИТМО>>}
		\end{center}
		
		\vspace{15em}
		
		\begin{center}
			\huge{\textbf{Расчётно-графическая работа №4}} \\
			\large{По дисциплине <<Математическая статистика>>}
		\end{center}
		
		\vspace{5em}
		
		\begin{flushright}
			\Large{\textbf{Михайлов Дмитрий Андреевич}} \\
			\Large{P3206} \\
			\Large{368530} \\
			\Large{\textbf{Медведев Владислав Александрович}} \\
			\Large{P3206} \\
			\Large{368508}
		\end{flushright}
		
		\vspace{10em}
		
		\begin{center}
			Санкт-Петербург \\
			2025 год
		\end{center}
	\end{titlepage}
	
	\tableofcontents
	\newpage
	
	\addcontentsline{toc}{section}{Задача №1}
	\section*{Задача №1}
	
	\textbf{Условие задачи.}
	
	Задание представлено в 4 вариантах. Для каждого варианта требуется построить линейную модель (предполагая нормальность распределения ошибок, некоррелированность компонент, гомоскедастичность), вычислить оценки коэффициентов модели и остаточной дисперсии, построить для них доверительные интервалы, вычислить коэффициент детерминации, проверить указанные в условии гипотезы с помощью построенной линейной модели.
	
	\textbf{Указание}: из встроенных функций разрешается пользоваться квантильными функциями и средствами для квадратичной оптимизации (иными словами, готовую обертку для построения линейной модели не использовать, максимум можете сравнить вашу реализацию с готовой)
	
	\textbf{Вариант 1}.
	В файле \href{https://drive.google.com/file/d/1vv2jGNp6EO8HHRoscDRQU90faR3j8iTN/view}{cars93.csv} представлены данные о продажах различных авто.
	
	\begin{enumerate}
		\item Постройте линейную модель, где в качестве независимых переменных выступают расход в городе, расход на шоссе, мощность (вместе со свободным коэффициентом), зависимой - цена.
		
		\item Проверьте следующие подозрения:
		\begin{itemize}
			\item Чем больше мощность, тем больше цена
			
			\item Цена изменяется в зависимости от расхода в городе
			
			\item Проверьте гипотезу $H_0$ о равенстве одновременно нулю коэффициентов при расходе в городе и расходе на шоссе против альтернативы $H_1 = \bar{H_0}$
		\end{itemize}
	\end{enumerate}
	
	\textbf{Решение.}
	
	Построение линейной модели с зависимой переменной Price и независимыми переменными:
	
	\begin{itemize}
		\item MPG.city (расход в городе)
		\item MPG.highway (расход на шоссе)
		\item Horsepower (мощность)
		\item Свободный коэффициент (константа)
	\end{itemize}
	\vspace*{1em}
	
	Используем метод наименьших квадратов.
	
	$$ \beta = (X^T X)^{-1} X^T y $$
	
	После нахождения коэффициентов $ \beta $ предположение делается как $ \hat{y} = X \beta $.
	
	Имеем следующее уравнение:
	
	$$ \text{Price} = \beta_0 + \beta_1 \cdot \text{MPG.city} + \beta_2 \cdot \text{MPG.highway} + \beta_3 \cdot \text{Horsepower} + \epsilon $$
	
	где:
	\begin{itemize}
		\item $\beta_0$ — свободный коэффициент (intercept)
		\item $\beta_1, \beta_2, \beta_3$ — коэффициенты при соответствующих переменных
		\item $\epsilon$ — ошибка (остатки модели)
	\end{itemize}
	
	Метод наименьших квадратов (МНК) чувствителен к выбросам, так как он минимизирует сумму квадратов отклонений, и даже один выброс может сильно исказить коэффициенты регрессии.
	\vspace*{1em}
	
	\textbf{Вывод.}
	
	После написания кода коэффициенты statsmodels и моей реализации практически полностью совпадают => модель построена верно.
	\vspace*{1em}
	
	\textbf{Первое подозрение.}
	
	\begin{itemize}
		\item Нулевая гипотеза $H_0$: Мощность не влияет на цену, т.е. коэффициент при переменной Horsepower равен 0:
		
		$$ H_0: \beta_{\text{horsepower}} = 0 $$
		
		\item Альтернативная гипотеза $H_1$: Чем выше мощность, тем выше цена, т.е. коэффициент положительный:
		
		$$ H_1: \beta_{\text{horsepower}} > 0 $$
	\end{itemize}
	
	Это односторонний t-тест на значимость одного коэффициента.
	
	Построена линейная модель зависимости цены автомобиля (Price) от:
	
	\begin{itemize}
		\item расхода топлива в городе (MPG.city)
		\item расхода на шоссе (MPG.highway)
		\item мощности двигателя (Horsepower)
		\item и свободного коэффициента (intercept)
	\end{itemize}
	
	Коэффициенты модели вычислены вручную через формулу нормального уравнения:
	
	$$ \hat{\beta} = (X^T X)^{-1} X^T y $$
	
	Затем была рассчитана стандартная ошибка коэффициента при Horsepower и t-статистика по формуле:
	
	$$ t = \frac{\hat{\beta}_{\text{horsepower}}}{SE(\hat{\beta}_{\text{horsepower}})} $$
	
	\textbf{Вывод.}
	
	После выполнения всех расчётов вручную получено:
	
	\begin{itemize}
		\item Коэффициент при Horsepower: 0.1313
		\item Стандартная ошибка: 0.0161
		\item t-статистика: 8.1530
		\item Критическое значение t для уровня значимости $\alpha = 0.05$ и степеней свободы $n - p$:
		
		$$ t_{\text{кр}} = t_{1-\alpha}(n - p) $$
	\end{itemize}
	
	Если наблюдаемое значение t-статистики превышает критическое значение:
	
	\begin{itemize}
		\item Мы отвергаем нулевую гипотезу $H_0$
		\item Значит, мощность статистически значимо влияет на цену, причём влияние положительное
	\end{itemize}
	
	Решение представлено на языке Python.
	\vspace*{1em}
	
	\textbf{Второе подозрение.}
	
	\begin{itemize}
		\item Нулевая гипотеза $H_0$: Расход в городе не влияет на цену. То есть коэффициент при MPG.city равен нулю:
		
		$$ H_0: \beta_{\text{MPG.city}} = 0 $$
		
		\item Альтернативная гипотеза $H_1$: Расход в городе влияет на цену, т.е. коэффициент отличен от нуля (двусторонняя альтернатива):
		
		$$ H_1: \beta_{\text{MPG.city}} \ne 0 $$
	\end{itemize}
	
	У нас есть наша линейная модель:
	
	$$
	\text{Price} = \beta_0 + \beta_1 \cdot \text{MPG.city} + \beta_2 \cdot \text{MPG.highway} + \beta_3 \cdot \text{Horsepower} + \varepsilon
	$$
	
	Для проверки значимости коэффициента $\beta_1$ (при MPG.city), мы:
	
	\begin{enumerate}
		\item рассчитываем его оценку $\hat{\beta}_1$
		\item определяем стандартную ошибку
		\item вычисляем t-статистику
		\item сравниваем с критическим значением t для двухстороннего теста при $\alpha = 0.05$
	\end{enumerate}
	
	\begin{itemize}
		\item Если наблюдаемое $|t|$ больше критического — гипотеза $H_0$ отвергается, расход в городе влияет на цену.
		
		\item Если $|t| \le t_{\text{кр}}$ — гипотеза $H_0$ не отвергается, доказательств влияния нет.
		
		\item Коэффициент при MPG.city равен $\hat{\beta}_1 = -0.0386$
		
		\item t-статистика: -0.1081
		
		\item Критическое значение t при уровне значимости 5\% (двусторонний тест): +-1.9870
	\end{itemize}
	
	Так как $|t|$ меньше критического значения, мы не отвергаем нулевую гипотезу.
	
	Таким образом, нет статистически значимых оснований утверждать, что расход топлива в городе влияет на цену автомобиля.
	\vspace*{2em}
	
	\textbf{Третье подозрение.}
	
	\begin{itemize}
		\item Нулевая гипотеза $H_0$:
		
		$$ \beta_{\text{MPG.city}} = 0,\quad \beta_{\text{MPG.highway}} = 0 $$
		
		\item Альтернативная гипотеза $H_1$: хотя бы один из коэффициентов не равен нулю
	\end{itemize}
	
	Это многомерная гипотеза, проверяется с помощью F-критерия сравнивая две модели:
	
	Полная модель: $$
	\text{Price} = \beta_0 + \beta_1 \cdot \text{MPG.city} + \beta_2 \cdot \text{MPG.highway} + \beta_3 \cdot \text{Horsepower} + \varepsilon
	$$
	
	Упрощённая модель ($H_0$): $$
	\text{Price} = \beta_0 + \beta_3 \cdot \text{Horsepower} + \varepsilon
	$$
	
	Мы проверяем гипотезу о том, что коэффициенты при переменных MPG.city и MPG.highway одновременно равны нулю.
	
	Построены две модели:
	\begin{itemize}
		\item Полная: с переменными MPG.city, MPG.highway и Horsepower
		
		\item Упрощённая: только с переменной Horsepower
	\end{itemize}
	
	Разница между остаточными суммами квадратов моделей (RSS) используется для расчёта F-статистики:
	
	$$ F = \frac{(RSS_{reduced} - RSS_{full}) / q}{RSS_{full} / (n - p)} $$
	
	Сравнивая F-статистику с критическим значением из F-распределения при уровне значимости $\alpha = 0.05$, получаем:
	
	\begin{itemize}
		\item Если $F > F_{crit}$, отвергаем $H_0$: влияние расхода есть
		
		\item Если $F \le F_{crit}$, не отвергаем $H_0$: расход топлива не влияет на цену автомобиля
	\end{itemize}
	
	После получения остаточных сумм квадратов (RSS) и F-статистики мы сравниваем её с критическим значением распределения Фишера при уровне значимости $\alpha = 0.05$.
	
	Если F-статистика превысила критическое значение, мы отвергли нулевую гипотезу, что означает:
	
	\textbf{Вывод.}
	
	F-статистика оказалась меньше критического значения, то нет статистических оснований утверждать, что расход топлива влияет на цену автомобиля. Решение представлено на языке Python.
	\newpage
	
	\addcontentsline{toc}{section}{Задача №2}
	\section*{Задача №2}
	
	\textbf{Условие задачи.}
	
	Для каждого варианта требуется проверить гипотезу о равенстве средних на каждом уровне фактора с помощью модели однофакторного дисперсионного анализа.
	
	\textbf{Указание}: реализовать самим.
	
	\textbf{Вариант 1.}
	В файле \href{https://drive.google.com/file/d/1CSCheMzjberRwgcf90BBu-J6uxMg-Qf7/view}{
	iris.csv} представлены данные об ирисках. Фактор - подвид. Выходная переменная - суммарная площадь (точнее оценка площади) чашелистика и лепестка.
	\vspace*{1em}
	
	\textbf{Решение.}
	
	Проверить, влияет ли категориальный фактор (подвид) на значение некоторого количественного признака (суммарная площадь чашелистника и лепестка)
	
	Это делается путём проверки нулевой гипотезы:
	
	$$ H_0: \mu_1 = \mu_2 = \cdots = \mu_k $$
	
	где $\mu_i$ — среднее значение признака в группе с $i$-м уровнем фактора, $k$ — количество уровней.
	
	Альтернативная гипотеза $H_1$: средние хотя бы двух групп различаются.
	
	Пусть:
	\begin{itemize}
		\item $X_{ij}$ — значение количественного признака для наблюдения $j$-го в $i$-й группе
		\item $n_i$ — число наблюдений в $i$-й группе
		\item $N = \sum_{i=1}^k n_i$ — общее число наблюдений
		\item $\bar{X}_i$ — среднее значение признака в $i$-й группе
		\item $\bar{X}$ — общее среднее значение по всем наблюдениям
	\end{itemize}
	
	\textbf{Расчётные формулы.}
	
	\begin{itemize}
		\item Общее среднее:
		
		$$ \bar{X} = \frac{1}{N} \sum_{i=1}^k \sum_{j=1}^{n_i} X_{ij} $$
		
		\item Межгрупповая сумма квадратов (SSB):
		
		$$ SSB = \sum_{i=1}^k n_i (\bar{X}_i - \bar{X})^2 $$
		
		Показывает, насколько отличаются средние групп от общего среднего — то есть, влияние фактора.
		
		\item Внутригрупповая сумма квадратов (SSW):
		
		$$ SSW = \sum_{i=1}^k \sum_{j=1}^{n_i} (X_{ij} - \bar{X}_i)^2 $$
		
		Показывает естественную дисперсию внутри групп — обусловленную случайными колебаниями.
	\end{itemize}
	
	\textbf{Статистика F.}
	
	\begin{itemize}
		\item Число степеней свободы:
		$df_b = k - 1$ — межгрупповое, $df_w = N - k$ — внутригрупповое
		
		\item Средние квадраты: $MSB = \frac{SSB}{df_b}$, $MSW = \frac{SSW}{df_w}$
	\end{itemize}
	
	Общая формула F-статистики.
	
	$$ F = \frac{MSB}{MSW} $$
	
	\textbf{Критерий принятия решения.}
	
	Сравниваем вычисленную F-статистику с критическим значением из распределения Фишера:
	
	$$ F > F_{\text{crit}}(\alpha; df_b, df_w) \Rightarrow \text{отклоняем } H_0 $$
	
	где $\alpha$ — уровень значимости (0.05).
	Критическое значение находится по квантильной функции:
	
	$$ F_{\text{crit}} = F^{-1}(1 - \alpha) $$
	
	с параметрами $df_b, df_w$.
	\vspace*{1em}
	
	\textbf{Принятие решения.}
	
	\begin{itemize}
		\item Если $F \leq F_{\text{crit}}$: фактор не оказывает значимого влияния — средние статистически одинаковы.
		
		\item Если $F > F_{\text{crit}}$: есть значимые различия между средними в разных группах — фактор влияет на результат.
	\end{itemize}
	
	Решение представлено на языке Python.
	\newpage
	
	\addcontentsline{toc}{section}{Приложения}
	\section*{Приложения}
	
	\subsection*{Задача №1}
	
	Ссылка на исходник с кодом программы, решающей эту задачу на языке Python. \cite{TaskNumber1}
	
	\subsection*{Задача №2}
	
	Ссылка на исходник с кодом программы, решающей эту задачу на языке Python. \cite{TaskNumber2}
	\newpage
	
	\addcontentsline{toc}{section}{Список использованных источников}
	\begin{thebibliography}{99}
		\bibitem{TaskNumber1}
		Задача №1. \textit{URL}: \href{https://colab.research.google.com/drive/1hU3V-uFTl9Eqq9hKGlbg4017R-LNO2Ed?usp=sharing}{Исходник с кодом, решающий задачу №1.}
		\bibitem{TaskNumber2}
		Задача №2. \textit{URL}: \href{https://colab.research.google.com/drive/1jZ2odN8TAdEDsBw6dNhFYbSYFFkQi2u0?usp=sharing}{Исходник с кодом, решающий задачу №2.}
	\end{thebibliography}
	
\end{document}